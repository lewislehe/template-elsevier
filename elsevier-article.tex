
\documentclass[
  preprint, %preprint style vs. review or final
  3p, %3p size
  times, %times font (delete to get computer modern)
  11pt, %font size
  authoryear %if you want harvard style authoryear citations. otherwise leave blank.
]{elsarticle}


%% Use the option review to obtain double line spacing
%% \documentclass[authoryear,preprint,review,12pt]{elsarticle}

%% Use the options 1p,twocolumn; 3p; 3p,twocolumn; 5p; or 5p,twocolumn
%% for a journal layout:
%% \documentclass[final,1p,times,authoryear]{elsarticle}
%% \documentclass[final,3p,times,authoryear]{elsarticle}
%% \documentclass[final,3p,times,twocolumn,authoryear]{elsarticle}
%% \documentclass[final,5p,times,authoryear]{elsarticle}

\usepackage{amssymb}
\usepackage{amsmath}
\usepackage{amsthm}

%% The lineno packages adds line numbers. Start line numbering with
%% \begin{linenumbers}, end it with \end{linenumbers}. Or switch it on
%% for the whole article with \linenumbers.
%% \usepackage{lineno}

\journal{Journal of Published Research}

\begin{document}

\begin{frontmatter}


  %%Research highlights
  % \begin{highlights}
  %   \item Research highlight 1
  %   \item Research highlight 2
  % \end{highlights}

  %% Title, authors and addresses

  %% use the tnoteref command within \title for footnotes;
  %% use the tnotetext command for theassociated footnote;
  %% use the fnref command within \author or \affiliation for footnotes;
  %% use the fntext command for theassociated footnote;
  %% use the corref command within \author for corresponding author footnotes;
  %% use the cortext command for theassociated footnote;
  %% use the ead command for the email address,
  %% and the form \ead[url] for the home page:
  %% \title{Title\tnoteref{label1}}
  %% \tnotetext[label1]{}
  %% \author{Name\corref{cor1}\fnref{label2}}
  %% \ead{email address}
  %% \ead[url]{home page}
  %% \fntext[label2]{}
  %% \cortext[cor1]{}
  %% \affiliation{organization={},
  %%            addressline={}, 
  %%            city={},
  %%            postcode={}, 
  %%            state={},
  %%            country={}}
  %% \fntext[label3]{}

  % \title{Elsevier article template\tnoteref{t1}} %% Article title




  \title{Title of the paper\tnoteref{t1}}

  \tnotetext[t1]{Research project funded by Skynet Foundation for the Humanities Grant A32-45.}


  %% use optional labels to link authors explicitly to addresses:
  \author[author1]{Leroy Jenkins\corref{cor1}}
  \affiliation[author1]{organization={University of Anywhere},
    addressline={123 Anywhere St},
    city={Anytown},
    postcode={12345},
    state={CA},
    country={USA}}

  \ead{ljenkins@example.com}
  \ead[url]{https://app.crixet.com}
  \cortext[cor1]{Corresponding author}


  \author[author2]{Parker Parallel\fnref{fn1}} %% Author name
  \ead{ljenkins@example.com}
  \fntext[fn1]{Parker Parallel disavows this work.}

  %% Author affiliation
  \affiliation[author2]{organization={University of Nowhere},%Department and Organization
    addressline={456 Nowhere Rd},
    city={Nowhere},
    postcode={67890},
    state={NY},
    country={USA}}

  %% Abstract
  \begin{abstract}
    %% Text of abstract
    Abstract text.
  \end{abstract}

  %%Graphical abstract
  % \begin{graphicalabstract}
  % \includegraphics{grabs}
  % \end{graphicalabstract}


  %% Keywords
  \begin{keyword}

    keyword \sep keyword

    %% PACS codes here, in the form: \PACS code \sep code

    %% MSC codes here, in the form: \MSC code \sep code
    %% or \MSC[2008] code \sep code (2000 is the default)

  \end{keyword}

\end{frontmatter}

%% Add \usepackage{lineno} before \begin{document} and uncomment 
%% following line to enable line numbers
%% \linenumbers

%% Use \section commands to start a section
\section{Example Section}
\label{sec1}
%% Labels are used to cross-reference an item using \ref command.

This tempalte uses the \verb1elsarticle.cls1 class file.
Check out \verb1documentation.pdf1 and comments in the code for more details. See Subsection \ref{subsec1}. Example in-text citation: \citet{Lehe2024}. Example parenthetical citation: \citep{Lehe2024}. At the end of the document you can choose a bibliography style by choosing
\begin{itemize}
  \item  \verb1\bibliographystyle{elsarticle-harv}1 for author year citations (harvard style) like \citet{Lehe2024}
  \item \verb1\bibliographystyle{elsarticle-num}1 for numeric citations  
  \item  \verb1\bibliographystyle{elsarticle-num-names}1  for names with numeric citations
\end{itemize}



%% Use \subsection commands to start a subsection.
\subsection{Example Subsection}
\label{subsec1}

Subsection text.

\subsubsection{Mathematics}
Math is done in the normal way.

\begin{equation}
  f(x) = (x+a)(x+b)
\end{equation}

\begin{align}
  f(x) & = (x+a)(x+b)        \\
       & = x^2 + (a+b)x + ab
\end{align}

\begin{eqnarray}
  f(x) &=& (x+a)(x+b) \nonumber\\ %% If equation numbering is not needed for a row use \nonumber.
  &=& x^2 + (a+b)x + ab
\end{eqnarray}



\begin{table}[t]%% placement specifier
  \centering%% For centre alignment of tabular.
  \begin{tabular}{l c r}%% Table column specifiers
    %% Tabular cells are separated by &
    1 & 2 & 3 \\ %% A tabular row ends with \\
    4 & 5 & 6 \\
    7 & 8 & 9 \\
  \end{tabular}
  \caption{Table Caption}\label{table1}
\end{table}


%% Use figure environment to create figures
%% Refer following link for more details.
%% https://en.wikibooks.org/wiki/LaTeX/Floats,_Figures_and_Captions
\begin{figure}[t]%% placement specifier
  \centering%% For centre alignment of image.
  \includegraphics{example-image-a}
  \caption{Figure Caption}\label{fig1}
\end{figure}


%% The Appendices part is started with the command \appendix;
%% appendix sections are then done as normal sections
\appendix
\section{Example Appendix Section}
\label{app1}

Appendix text.

%% For citations use: 
%%       \citet{<label>} ==> Lamport (1994)
%%       \citep{<label>} ==> (Lamport, 1994)
%%

%% If you have bib database file and want bibtex to generate the
%% bibitems, please use
%%
\bibliographystyle{elsarticle-num-names}
\bibliography{BibFile}

%% else use the following coding to input the bibitems directly in the
%% TeX file.

%% Refer following link for more details about bibliography and citations.
%% https://en.wikibooks.org/wiki/LaTeX/Bibliography_Management

% You can use the following code to input the bibitems directly in the  
% TeX file.
% \begin{thebibliography}{00}

%   %% For authoryear reference style
%   %% \bibitem[Author(year)]{label}
%   %% Text of bibliographic item

%   \bibitem[Lamport(1994)]{lamport94}
%   Leslie Lamport,
%   \textit{\LaTeX: a document preparation system},
%   Addison Wesley, Massachusetts,
%   2nd edition,
%   1994.

% \end{thebibliography}
\end{document}

% \endinput
%%


